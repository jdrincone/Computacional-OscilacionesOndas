%#########################################################################
\chapter{Reflexión, Refracción, Óptica}
\label{cha:optic}


%#########################################################################



%*************************************************************************
\section{Demostración 1: Interferencia de ondas en una superficie}
\label{sec:DEMO4_01}
\rule{14cm}{0.5mm}

La interferencia es un fenómeno que se presenta en movimientos ondulatorios
y consiste en la superposición de campos, ya sea de forma constructiva o 
destructiva. En esta demostración se calcula la interferencia de dos fuentes
puntuales sobre una superficie plana (por ejemplo un fluido).


\
%.........................................................................
%Doppler effect
\begin{figure}[htbp]
	\centering
	\includegraphics[width=0.60\textwidth]
	{./pictures/interference.png}

	\caption{\small{Dos fuentes puntuales sobre un fluido.}}
	
	\label{fig:doppler}
\end{figure}
%.........................................................................


Por simplicidad se describirá la oscilación producida por las fuentes como
una onda radial de la forma

%.........................................................................
%Radial wave
\eq{eq:radial_wave}
{z(r, t) = A_0\sin\pr{\omega t - k r } = A_0\sin\cor{2\pi \pr{\frac{t}{f} - \frac{r}{\lambda} }}}
%.........................................................................

donde $z(r, t)$ es la función de amplitud en $z$ de la onda, $A_0$ la 
amplitud máxima del desplazamiento, $\omega$ la frecuencia angular, $f$ la 
frecuencia y $\lambda$ la longitud de onda. El valor de $r$ es medido desde 
el origen de la perturbación, es decir, en la posición de cada fuente.
Para calcular la amplitud total se usa el principio de superposición, donde
se suman las contribuciones de cada fuente.

\

En orden para correr el código de la demostración apropiadamente, es 
necesario instalar el siguiente programa

%.........................................................................
%Installing pyaudio
\begin{listing}[style=consola, numbers=none]
\$ sudo apt-get install ffmpeg
\end{listing}
%.........................................................................

Este programa permite generar videos a partir de un conjunto de imágenes
nombradas apropiadamente.

\

A continuación de ilustra el código de la demostración


%ccccccccccccccccccccccccccccccccccccccccccccccccccccccccccccccccccccccccc
%DEMO 4_01
\begin{listing}[style=python]
# -*- coding: utf-8 -*-
#!/usr/bin/env python
#==========================================================
# DEMOSTRACION 1
# Interferencia de Ondas
#==========================================================

#**********************************************************
#	MODULOS
#**********************************************************
import numpy as np
import os
import matplotlib.pylab as plt

#Funcion de amplitud de una fuente
def Z_amplitud(x, y, x0, y0, A0, t):
    #Posicion de la fuente
    r0 = np.array([x0, y0])
    #Posicion donde se evalua el campo
    r = np.array([x, y])
    return A0*np.sin( 2*np.pi*(t/f - np.linalg.norm(r-r0)/lamb ) )

#PARAMETROS
#frecuencia		[Hz]
f = 10.
#Longitud de onda	[m]
lamb = 1.0
#Tamano de la caja	[m]
L = 10.0
#Resolucion grid caja
NS = 100

#Posicion X fuente 1	[m]
x01 = 1.0	
#Posicion Y fuente 1	[m]
y01 = 2.0
#Amplitud fuente 1	[m]
A01 = 0.1
#Posicion X fuente 2	[m]
x02 = 5.0	
#Posicion Y fuente 2	[m]
y02 = 5.0
#Amplitud fuente 2	[m]
A02 = 0.1

#Tiempo final		[s]
Tmax = 20.0
#Numero de intervalos
Nstep = 41

#GRID Y EVOLUCION
XM = np.linspace( 0, L, NS )
YM = np.linspace( 0, L, NS )

k = 0
for t in np.linspace( 0, Tmax, Nstep ):
    Z = np.zeros( (NS,NS) )
    for i in xrange( 0, NS ):
	for j in xrange( 0, NS ):
	    x = XM[i]
	    y = YM[j]
	    Z[-j,i] = Z_amplitud(x, y, x01, y01, A01, t) + \
	    Z_amplitud(x, y, x02, y02, A02, t)
	    
    plt.xlabel( 'X (0,L)' )
    plt.ylabel( 'Y (0,L)' )
    plt.title( 'Interfencia: t=%f'%(t) )
    plt.imshow( Z, extent = (0,L,0,L), vmax = A01 + A02 )
    print t
    fname='_tmp-%03d.png'%k
    k+=1
    plt.savefig(fname)
    plt.close()

print 'Making movie animation.mpg - this make take a while'
os.system("ffmpeg -f image2 -i _tmp-%03d.png  video.mpg")
os.system('rm -rf *.png')
\end{listing}
%ccccccccccccccccccccccccccccccccccccccccccccccccccccccccccccccccccccccccc

El resultado obtenido es un video donde se ilustra la evolución del patrón
de interferencia. Una toma del video es realizada en la siguiente figura


%.........................................................................
%Doppler signal
\begin{figure}[htbp]
	\centering
	\includegraphics[width=0.90\textwidth]
	{./pictures/Two_Sources.png}

	\caption{\small{Patrón de interferencia de dos fuentes.}}
	
	\label{fig:interference}
\end{figure}
%.........................................................................


A continuación se explica cada parte del código


%ccccccccccccccccccccccccccccccccccccccccccccccccccccccccccccccccccccccccc
%DEMO 4_01
\begin{listing}[style=python, numbers = none]
import numpy as np
import os
import matplotlib.pylab as plt
\end{listing}
%ccccccccccccccccccccccccccccccccccccccccccccccccccccccccccccccccccccccccc
Se cargan las librerías estándar.


%ccccccccccccccccccccccccccccccccccccccccccccccccccccccccccccccccccccccccc
%DEMO 4_01
\begin{listing}[style=python, numbers = none]
#Funcion de amplitud de una fuente
def Z_amplitud(x, y, x0, y0, A0, t):
    #Posicion de la fuente
    r0 = np.array([x0, y0])
    #Posicion donde se evalua el campo
    r = np.array([x, y])
    return A0*np.sin( 2*np.pi*(t/f - np.linalg.norm(r-r0)/lamb ) )
\end{listing}
%ccccccccccccccccccccccccccccccccccccccccccccccccccccccccccccccccccccccccc
Se define la función de amplitud para una fuente puntual. Los argumentos
son, la coordenada del punto donde se desea evaluar el campo (\texttt{x},
\texttt{y}), la coordenada de la fuente (\texttt{x0}, \texttt{y0}), la
amplitud de la oscilación \texttt{A0} y el tiempo actual \texttt{t}.


%ccccccccccccccccccccccccccccccccccccccccccccccccccccccccccccccccccccccccc
%DEMO 4_01
\begin{listing}[style=python, numbers = none]
#PARAMETROS
#frecuencia		[Hz]
f = 10.
#Longitud de onda	[m]
lamb = 1.0
#Tamano de la caja	[m]
L = 10.0
#Resolucion grid caja
NS = 100

#Posicion X fuente 1	[m]
x01 = 1.0	
#Posicion Y fuente 1	[m]
y01 = 2.0
#Amplitud fuente 1	[m]
A01 = 0.1
#Posicion X fuente 2	[m]
x02 = 5.0	
#Posicion Y fuente 2	[m]
y02 = 5.0
#Amplitud fuente 2	[m]
A02 = 0.1

#Tiempo final		[s]
Tmax = 20.0
#Numero de intervalos
Nstep = 41

#GRID Y EVOLUCION
XM = np.linspace( 0, L, NS )
YM = np.linspace( 0, L, NS )
\end{listing}
%ccccccccccccccccccccccccccccccccccccccccccccccccccccccccccccccccccccccccc
Se definen todos los parámetros del sistema, incluyendo el grid donde se 
evaluará el campo de desplazamiento de la onda.


%ccccccccccccccccccccccccccccccccccccccccccccccccccccccccccccccccccccccccc
%DEMO 4_01
\begin{listing}[style=python, numbers = none]
k = 0
for t in np.linspace( 0, Tmax, Nstep ):
    Z = np.zeros( (NS,NS) )
    for i in xrange( 0, NS ):
	for j in xrange( 0, NS ):
	    x = XM[i]
	    y = YM[j]
	    Z[-j,i] = Z_amplitud(x, y, x01, y01, A01, t) + \
	    Z_amplitud(x, y, x02, y02, A02, t)

    plt.xlabel( 'X (0,L)' )
    plt.ylabel( 'Y (0,L)' )
    plt.title( 'Interfencia: t=%f'%(t) )
    plt.imshow( Z, extent = (0,L,0,L), vmax = A01 + A02 )
    print t
    fname='_tmp-%03d.png'%k
    k+=1
    plt.savefig(fname)
    plt.close()
\end{listing}
%ccccccccccccccccccccccccccccccccccccccccccccccccccccccccccccccccccccccccc
Se realiza un ciclo \texttt{for} para la iteración de cada tiempo del 
sistema. Se hace un barrido sobre las dos coordenadas \texttt{x = XM[i]},
\texttt{y = YM[j]} y se calcula el valor del campo de las dos fuentes 
\texttt{Z[-j,i] = Z\_amplitud(x, y, x01, y01, A01, t) + 
Z\_amplitud(x, y, x02, y02, A02, t)}. Usando la función 
\texttt{plt.savefig(fname)} de \matplotlib, se guarda la grafica del campo
en el tiempo actual, con el nombre \texttt{'\_tmp-\%03d.png'\%k} donde 
\texttt{k} es un contador entero. Una vez guardada la gráfica, se cierra
el entorno gráfico actual con la función \texttt{plt.close()} y se procede
a calcular el siguiente tiempo.



%ccccccccccccccccccccccccccccccccccccccccccccccccccccccccccccccccccccccccc
%DEMO 4_01
\begin{listing}[style=python, numbers = none]
os.system("ffmpeg -f image2 -i _tmp-%03d.png  video.mpg")
os.system('rm -rf *.png')
\end{listing}
%ccccccccccccccccccccccccccccccccccccccccccccccccccccccccccccccccccccccccc
Finalmente se realiza el video a partir del programa \texttt{ffmpeg}. El 
formato de uso puede ser consultado en la página oficial del proyecto 
\footnote{\url{http://www.ffmpeg.org/}}, aunque los parámetros usados en
este código son óptimos para la generación de videos en \python.

%*************************************************************************



\newpage
%*************************************************************************
\section{Ejercicios}
\label{sec:ejercicios}

%.........................................................................
\subsection*{Ejercicio 1 \large{$\pr{\star}$}}

\textbf{Fuente puntual con reflexión}

Considere el mismo sistema de la demostración 1, pero con una fuente puntual. 
%Reflection
\begin{figure}[htbp]
	\centering
	\includegraphics[width=0.60\textwidth]
	{./pictures/interference_reflex.png}

	\caption{\small{Reflexión en paredes.}}
	
	\label{fig:wall_reflection}
\end{figure}

A partir de este sistema, hacer un código que realice un video donde se
muestre el patrón obtenido en el tiempo debido a la reflexión en las paredes 
de la caja. 

\

\textit{Hint:} para simular la reflexión de las ondas, puede considerar 
fuentes ficticias en el exterior de la caja que tengan posiciones simétricas
respecto a las fuentes originales.
%.........................................................................


%.........................................................................
\subsection*{Ejercicio 2 \large{$\pr{\star}$}}

\textbf{Refracción de sonido}

Considere una onda sonora monocromática (sinusoidal) que se propaga en el 
aire, usando la librería \texttt{AudioLib} \footnote{ver ejemplo de uso en 
demostración 2 del capítulo 3 (Efecto Dopler)} simule el audio que sería 
percibido por un observador en el mismo medioo (aire) y otro audio por un
observador que está sumergido en una piscina (agua). Es decir, simule el 
cambio que sufre el sonido por refracción en el agua.

\

\textit{Hint:} todos los valores de los parámetros son de libre elección,
la única condición es coherencia física en el resultado. La función 
\texttt{savewav} de \texttt{AudioLib} puede ser de utilidad (en ipython,
después de importar la librería y crear un objeto \texttt{audio}, 
escribir \texttt{<object\_name>.savewav?}).
%.........................................................................

\newpage
%.........................................................................
\subsection*{Ejercicio 3 \large{$\pr{\star}$}}

\textbf{Simulación de trayectoria de luz en una lente}

Considere un conjunto de $N$ rayos de luz que atraviesan una de las lentes
de la siguiente figura.

%Lens system
\begin{figure}[htbp]
	\centering
	\includegraphics[width=0.30\textwidth]
	{./pictures/lens.png}

	\caption{\small{Efecto de una lente en haces de luz.}}
	
	\label{fig:lens}
\end{figure}

Realice un código que grafique (2D) la desviación de los haces de luz para 
cada uno de los dos casos (El usario del código debería poder escoger cual 
de los dos lentes usar). Los parámetros físicos son de libre elección, 
incluyendo el número de haces iniciales y la separación entre ellos. 

\

\textit{Hint:} La forma de los lentes puede omitirse en el gráfico, tan 
solo se puede ilustrar como una línea recta, o una línea divisoria que
marque el punto donde los rayos se deflectan.\footnote{Un ejemplo algo más
complejo que ilustra la trayetoria de haces en lentes puede encontrarse
en \url{https://raw.github.com/sbustamante/Computacional-OscilacionesOndas/master/codigos/lens.py}}

%.........................................................................


\newpage
%.........................................................................
\subsection*{Ejercicio 4 \large{$\pr{\star}$}}

\textbf{Refracción de fuente puntual.}

Considere una fuente puntual en un medio con índice de refracción $n_1$ y 
un medio adyacente con índice $n_2$, de tal forma que la interfaz entre ambos
medios es plana. 

%Refraction system
\begin{figure}[htbp]
	\centering
	\includegraphics[width=0.60\textwidth]
	{./pictures/difraction.png}

	\caption{\small{Refracción en un medio.}}
	
	\label{fig:two_medias}
\end{figure}

A partir de este sistema, hacer un código que realice un video donde se
muestre el patrón obtenido en el tiempo debido a la refracción de un medio a
otro. 

\

\textit{Hint:} para simular la refracción de las ondas, puede considerar 
una fuente ficticia que produzca el patrón refractado, pero solo evaluarlo
en el medio 2.
%.........................................................................



\newpage
%.........................................................................
\subsection*{Ejercicio 5 \large{$\pr{\star \star}$}}

\textbf{Refracción Mecánica}

El fenómeno de refracción puede ser demostrado de manera totalmente mecánica
y se presenta sistemas como el descrito en la siguiente figura

%Mechanical Refraction
\begin{figure}[htbp]
	\centering
	\includegraphics[width=0.60\textwidth]
	{./pictures/soldiers.png}

	\caption{\small{Refracción mecánica.}}
	
	\label{fig:mechanical_refraction}
\end{figure}

Considere un pelotón de soldados marchando de forma sincronizada, de tal  
forma que siempre están alineados en frentes. El pelotón tiene una 
disposición de $N\times M$ soldados. Inicialmente están marchando en 
un terreno poco agreste (césped) y el pelotón se mueve a una velocidad
constante $v_0$. Más adelante en el camino, se encuentra un terreno 
pantanoso por el cual es más difícil transitar y el pelotón se mueve a 
una velocidad menor $v_1$. Si los soldados entran al terreno pantanoso
formando un ángulo $\theta$ con la normal de la interfaz de los terrenos.
Usando solo la condición de mantener los frentes alineados, puede 
demostrarse que el pelotón cambiará de orientación.

\

Basado en esta situación, realice un código donde se ilustre en una 
animación 3D este efecto.

\

\textit{Hint:} Por simplicidad, los soldados pueden ser tomados como cubos.
Puede usar la librería \mayavi o \textit{Vpython}.
%.........................................................................

%*************************************************************************
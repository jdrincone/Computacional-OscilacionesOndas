%#########################################################################
\chapter{Preliminares}
\label{cha:prem}
%#########################################################################


%*************************************************************************
\section{Motivación}
\label{sec:motiv}


La física ha evolucionado hasta un estado actual donde la mayoría de 
cálculos teóricos necesarios para realizar investigación de frontera 
requieren de una gran componente computacional. Desde la corroboración 
entre teoría y experimento, la predicción y control de los resultados de 
un experimento hecho a posteriori y la recreación de condiciones imposibles 
de lograr experimentalmente, tales como simulaciones cosmológicas del 
universo a gran escala o complejos sistemas atómicos. Estos son sólo 
algunos ejemplos representativos del papel de la computación en la física 
moderna. Debido a esto, el principal objetivo del suplemento computacional 
es la introducción temprana en los cursos de física básica de herramientas 
computacionales que serán de utilidad a los estudiantes en este curso 
específico y durante el transcurso de sus carreras científicas.


%*************************************************************************




%*************************************************************************
\section{Instalación de Paquetes}
\label{sec:install}


En la totalidad de esta guía será usado el lenguaje de programación \python
como referente para todos las prácticas y ejercicios computaciones. La 
principal motivación de esto es su facilidad de implementación en 
comparación a otros lenguajes también de amplio en ciencia. Además es un 
lenguaje interpretado, lo que permite una depuración más sencilla por 
parte del estudiante, sin necesidad de usar más complicados sistemas de 
depuración en el caso de lenguajes compilados como C o Fortran. \python es 
un lenguaje de código abierto, lo que permite la libre distribución del 
paquete y evita el pago de costosas licencias de uso, además la gran 
mayoría de paquetes que extienden enormemente la funcionalidad de \python 
son también código abierto y de libre distribución y uso.

\

A pesar de que \python es un lenguaje multiplataforma, permitiendo correr 
scripts python en Linux, Windows y Mac, acá solo se indicará el método de 
instalación para distribuciones Linux basadas en Debian.

\

La última versión de \python de la rama 2 es 2.7.4 y de la rama 3 es la 
3.3.1, debido a ligeras incompatibilidades entre ambas ramas de desarrollo, 
será utilizada la rama 2 en una de sus últimas versiones. En orden, para 
instalar \python en una versión Linux basta con descargarlo directamente 
de los repositorios oficiales\footnote{En la mayoría de distribuciones Linux
\python viene precargado por defecto.}, en el caso de una distro basada 
en Debian el gestor de paquetes es \texttt{apt-get}, y desde una terminal 
se tiene


%.........................................................................
%Install python
\begin{listing}[style=consola, numbers=none]
\$ apt-get install python2.7
\end{listing}
%.........................................................................


también puede descargarse directamente desde la página oficial del proyecto 
\url{http://python.org/}.

\

Una vez instalada la última versión de \python, es necesario instalar los
siguiente paquetes para el correcto desarrollo de las aplicaciones del 
curso:


%-------------------------------------------------------------------------
%Ipython
\subsection*{iPython}

\ipython es un shell que permite una interacción más interactiva con los
scripts de python, permitiendo el resaltado de sintaxis desde consola, 
funciones de autocompletado y depuración de código más simple. Para su 
instalación basta descargarlo de los repositorios oficiales 


%.........................................................................
%Install ipython
\begin{listing}[style=consola, numbers=none]
\$ apt-get install ipython
\end{listing}
%.........................................................................


o puede de descargarse de la página oficial \url{http://ipython.org/}. 
También puede encontrarse documentación completa y actualizada en esta 
página, se recomienda visitarla frecuentemente para tener las más recientes 
actualizaciones.


%-------------------------------------------------------------------------




%-------------------------------------------------------------------------
%NumPy
\subsection*{NumPy}

\numpy es una librería que extiende las funciones matemáticas de \python, 
permitiendo el manejo de matrices y vectores. Es esencial para la 
programación científica en \python y puede ser instalada de los repositorios


%.........................................................................
%Install NumPy
\begin{listing}[style=consola, numbers=none]
\$ apt-get install python-numpy
\end{listing}
%.........................................................................


La última versión estable es la 1.6.2. En la página oficial del proyecto 
puede encontrarse versiones actualizadas y una amplia documentación 
\url{http://www.numpy.org/}.

%-------------------------------------------------------------------------




%-------------------------------------------------------------------------
%SciPy
\subsection*{SciPy}

\scipy es una amplia biblioteca de algoritmos matemáticos para \python, 
esta incluye herramientas que van desde funciones especiales, integración,
optimización, procesamiento de señales, análisis de Fourier, etc. Al igual
que los anteriores paquetes, puede ser instalada desde los repositorios 
oficiales


%.........................................................................
%Install SciPy
\begin{listing}[style=consola, numbers=none]
\$ apt-get install python-scipy
\end{listing}
%.........................................................................


Una completa documentación del paquete puede ser encontrada en 
\url{http://docs.scipy.org/doc/scipy/reference/}. La última versión estable
es la 0.11.0 y puede ser encontrada en la página oficial del proyecto 
\url{http://www.scipy.org/}.


%-------------------------------------------------------------------------




%-------------------------------------------------------------------------
%Matplotlib
\subsection*{Matplotlib}

\matplotlib es una completa librería con rutinas para la generación de 
gráficos a partir de datos. Aunque en su estado actual está enfocada 
principalmente a gráficos 2D, permite un amplio control sobre el formato 
de las gráficas generadas, dando una amplia versatilidad a los usuarios.
Su instalación puede realizarse a partir de los repositorios oficiales


%.........................................................................
%Install Matplotlib
\begin{listing}[style=consola, numbers=none]
\$ apt-get install python-matplotlib
\end{listing}
%.........................................................................


La última versión estable es la 1.2.1. y puede encontrarse en la página 
oficial del proyecto \url{http://matplotlib.org/}. Una amplia documentación
está disponible en \url{http://matplotlib.org/1.2.0/contents.html}.


%-------------------------------------------------------------------------



%-------------------------------------------------------------------------
%Mayavi
\subsection*{MayaVi2}

\mayavi es una librería para la visualización científica en python, en 
especial para gráficos 3D, permitiendo funciones avanzadas como renderizado,
manejo de texturas, etc. Se encuentra en los repositorios oficiales 


%.........................................................................
%Install mayavi2
\begin{listing}[style=consola, numbers=none]
\$ apt-get install mayavi2
\end{listing}
%.........................................................................


La versión 2 es una versión mejorada de la original, estando más orientada
a la reutilización de código. Por defecto incluye una interfaz gráfica que
facilita su manejo. La página oficial del proyecto es 
\url{http://mayavi.sourceforge.net/}.

%-------------------------------------------------------------------------





%-------------------------------------------------------------------------
%Tkinter
\subsection*{Tkinter}

\tkinter es una librería para la gestión gráfica de aplicaciones in \python 
y viene por defecto instalada, aún así puede ser instalada de los 
repositorios oficiales


%.........................................................................
%Install Tkinter
\begin{listing}[style=consola, numbers=none]
\$ apt-get install python-tk
\end{listing}
%.........................................................................


La página oficial del proyecto es \url{http://wiki.python.org/moin/TkInter}.
Para el desarrollo de entornos gráficos existen otras llamativas
alternativas como PyGTK o PyQt, pero debido a la facilidad de uso y a ser
la librería estándar soportada, \tkinter será usada en este curso.


%-------------------------------------------------------------------------


%*************************************************************************
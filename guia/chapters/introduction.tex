%#########################################################################
\chapter{Introducción}

El principal objetivo del suplemento computacional es la introducción 
temprana en los cursos de física básica de herramientas computacionales, 
estas serán de utilidad a los estudiantes en este curso específico y 
durante el transcurso de sus carreras.


La física ha evolucionado hasta un estado actual donde la mayoría de 
cálculos teóricos necesarios para realizar investigación de frontera 
requieren de una gran componente computacional. Desde la corroboración 
entre teoría y experimento, la predicción y control de los resultados de 
un experimento hecho a posteriori, la recreación de condiciones imposibles 
de lograr experimentalmente, tales como simulaciones cosmológicas del 
universo a gran escala o complejos sistemas atómicos. Estos son sólo 
algunos ejemplos representativos del papel de la computación en la física 
moderna.
%#########################################################################
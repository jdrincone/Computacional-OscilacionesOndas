\documentclass[10pt]{book}
\usepackage [spanish] {babel}
\usepackage [T1]{fontenc}
\usepackage [utf8]{inputenc}
\usepackage {graphicx}
\usepackage{color} 
\usepackage{anysize} 
\usepackage[bookmarks=true]{hyperref}

\marginsize{2cm}{1cm}{1cm}{2cm}


%MY COMMANDS
\newcommand{\sub}[1]{\mbox{\scriptsize{#1}}}
\newcommand{\der}[2]{ \frac{ \partial #1 }{\partial #2} }
\newcommand{\dtot}[2]{ \frac{ d #1 }{d #2} }
\newcommand{\pr}[1]{ \left( #1 \right) }
\newcommand{\cor}[1]{ \left[ #1 \right] }
\newcommand{\lla}[1]{ \left\{ #1 \right\} }
\newcommand{\eq}[2]{\begin{equation} \label{eq#1} #2 \end{equation}}


\begin{document}
\title{Suplemento Computacional \\
\begin{Huge}
\textbf{Física de Oscilaciones y Ondas}
\end{Huge}}
\author{ Sebastian Bustamante Jaramillo\\ \begin{small}
macsebas33@gmail.com
\end{small} \\ \vspace{5cm} \\
\includegraphics[width=3cm]{UdeA_Shield} \\
Facultad de Ciencias Exactas y Naturales \\ 
Universidad de Antioquia }
\date{}

\maketitle


\newpage{\pagestyle{empty}\cleardoublepage}  

\tableofcontents
\newpage{\pagestyle{empty}\cleardoublepage}  

\chapter{Introducción}

\chapter{Instalación}

\chapter{Módulos}
	\section{data.py}
	\section{mechanic.py}
	\section{thermal.py}
	\section{magnetic.py}
	\section{numeric.py}

\chapter{Bibliografía}
\end{document}
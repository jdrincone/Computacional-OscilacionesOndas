\documentclass[12pt]{book}
\usepackage [spanish] {babel}
\usepackage [T1]{fontenc}
\usepackage [utf8]{inputenc}
\usepackage {graphicx}
\usepackage{color} 
\usepackage{anysize} 
\usepackage[bookmarks=true]{hyperref}

\marginsize{3cm}{2cm}{2cm}{3cm}
\usepackage{times}


%MY COMMANDS
\newcommand{\sub}[1]{\mbox{\scriptsize{#1}}}
\newcommand{\der}[2]{ \frac{ \partial #1 }{\partial #2} }
\newcommand{\dtot}[2]{ \frac{ d #1 }{d #2} }
\newcommand{\pr}[1]{ \left( #1 \right) }
\newcommand{\cor}[1]{ \left[ #1 \right] }
\newcommand{\lla}[1]{ \left\{ #1 \right\} }
\newcommand{\eq}[2]{\begin{equation} \label{eq#1} #2 \end{equation}}


\begin{document}
\title{Suplemento Computacional \\
\begin{Huge}
\textbf{Física de Oscilaciones y Ondas}
\end{Huge}}
\author{ Sebastian Bustamante Jaramillo\\ \begin{small}
macsebas33@gmail.com
\end{small} \\ \vspace{5cm} \\
\includegraphics[width=3cm]{UdeA_Shield} \\
Facultad de Ciencias Exactas y Naturales \\ 
Universidad de Antioquia }
\date{}

\maketitle


\newpage{\pagestyle{empty}\cleardoublepage}  

\tableofcontents
\newpage{\pagestyle{empty}\cleardoublepage}  

\chapter{Introducción}

El principal objetivo del suplemento computacional es la introducción temprana en los 
cursos de física básica de herramientas computacionales, estas serán de utilidad a los 
estudiantes en este curso específico y durante el transcurso de sus carreras.


La física ha evolucionado hasta un estado actual donde la mayoría de cálculos teóricos 
necesarios para realizar investigación de frontera requieren de una gran componente 
computacional. Desde la corroboración entre teoría y experimento, la predicción y control 
de los resultados de un experimento hecho a posteriori, la recreación de condiciones 
imposibles de lograr experimentalmente, tales como simulaciones cosmológicas del universo 
a gran escala o complejos sistemas atómicos. Estos son sólo algunos ejemplos 
representativos del papel de la computación en la física moderna.


\chapter{Instalación}

\chapter{Módulos}
	\section{data.py}
	\section{mechanic.py}
	\section{thermal.py}
	\section{magnetic.py}
	\section{numeric.py}

\chapter{Bibliografía}
\end{document}